\documentclass[,]{article}
\usepackage{lmodern}
\usepackage{amssymb,amsmath}
\usepackage{ifxetex,ifluatex}
\usepackage{fixltx2e} % provides \textsubscript
\ifnum 0\ifxetex 1\fi\ifluatex 1\fi=0 % if pdftex
  \usepackage[T1]{fontenc}
  \usepackage[utf8]{inputenc}
\else % if luatex or xelatex
  \ifxetex
    \usepackage{mathspec}
  \else
    \usepackage{fontspec}
  \fi
  \defaultfontfeatures{Ligatures=TeX,Scale=MatchLowercase}
\fi
% use upquote if available, for straight quotes in verbatim environments
\IfFileExists{upquote.sty}{\usepackage{upquote}}{}
% use microtype if available
\IfFileExists{microtype.sty}{%
\usepackage{microtype}
\UseMicrotypeSet[protrusion]{basicmath} % disable protrusion for tt fonts
}{}
\usepackage[margin=1in]{geometry}
\usepackage{hyperref}
\hypersetup{unicode=true,
            pdftitle={Отчёт},
            pdfborder={0 0 0},
            breaklinks=true}
\urlstyle{same}  % don't use monospace font for urls
\ifnum 0\ifxetex 1\fi\ifluatex 1\fi=0 % if pdftex
  \usepackage[shorthands=off,main=]{babel}
\else
  \usepackage{polyglossia}
  \setmainlanguage[]{}
\fi
\usepackage{graphicx}
% grffile has become a legacy package: https://ctan.org/pkg/grffile
\IfFileExists{grffile.sty}{%
\usepackage{grffile}
}{}
\makeatletter
\def\maxwidth{\ifdim\Gin@nat@width>\linewidth\linewidth\else\Gin@nat@width\fi}
\def\maxheight{\ifdim\Gin@nat@height>\textheight\textheight\else\Gin@nat@height\fi}
\makeatother
% Scale images if necessary, so that they will not overflow the page
% margins by default, and it is still possible to overwrite the defaults
% using explicit options in \includegraphics[width, height, ...]{}
\setkeys{Gin}{width=\maxwidth,height=\maxheight,keepaspectratio}
\IfFileExists{parskip.sty}{%
\usepackage{parskip}
}{% else
\setlength{\parindent}{0pt}
\setlength{\parskip}{6pt plus 2pt minus 1pt}
}
\setlength{\emergencystretch}{3em}  % prevent overfull lines
\providecommand{\tightlist}{%
  \setlength{\itemsep}{0pt}\setlength{\parskip}{0pt}}
\setcounter{secnumdepth}{0}
% Redefines (sub)paragraphs to behave more like sections
\ifx\paragraph\undefined\else
\let\oldparagraph\paragraph
\renewcommand{\paragraph}[1]{\oldparagraph{#1}\mbox{}}
\fi
\ifx\subparagraph\undefined\else
\let\oldsubparagraph\subparagraph
\renewcommand{\subparagraph}[1]{\oldsubparagraph{#1}\mbox{}}
\fi

%%% Use protect on footnotes to avoid problems with footnotes in titles
\let\rmarkdownfootnote\footnote%
\def\footnote{\protect\rmarkdownfootnote}

%%% Change title format to be more compact
\usepackage{titling}

% Create subtitle command for use in maketitle
\providecommand{\subtitle}[1]{
  \posttitle{
    \begin{center}\large#1\end{center}
    }
}

\setlength{\droptitle}{-2em}

  \title{Отчёт}
    \pretitle{\vspace{\droptitle}\centering\huge}
  \posttitle{\par}
    \author{}
    \preauthor{}\postauthor{}
    \date{}
    \predate{}\postdate{}
  

\begin{document}
\maketitle

\begin{verbatim}
##    Min. 1st Qu.  Median    Mean 3rd Qu.    Max. 
##   5.020   5.330   5.430   5.448   5.540   5.850
\end{verbatim}

Дисперсия:

\begin{verbatim}
## [1] 0.02867475
\end{verbatim}

СКО:

\begin{verbatim}
## [1] 0.1693362
\end{verbatim}

Размах вариации:

\begin{verbatim}
## [1] 0.83
\end{verbatim}

Коэффициент вариации:

\begin{verbatim}
## [1] 0.03108227
\end{verbatim}

IQR:

\begin{verbatim}
## [1] 0.21
\end{verbatim}

Квартильное отклонение:

\begin{verbatim}
## [1] 0.105
\end{verbatim}

Относительный показатель квартильной вариации:

\begin{verbatim}
## [1] 1.933702
\end{verbatim}

Относительное линейное отклонение:

\begin{verbatim}
## [1] 4.844347
\end{verbatim}

Коэффициент асимметрии:

\begin{verbatim}
## [1] -0.02367625
\end{verbatim}

Коэффициент эксцесса:

\begin{verbatim}
## [1] -0.0517866
\end{verbatim}

Если округлить значения до целого, то получим моду -2. И среднее
значение -1.68.

Проверим гипотезу о нормальности распределения с помощью критерия
Пирсона.

\begin{verbatim}
## [1] "Границы интервалов:"
\end{verbatim}

\begin{verbatim}
##  Нижняя граница Верхняя граница 
##        4.957536        5.082464
\end{verbatim}

\begin{verbatim}
##  Нижняя граница Верхняя граница 
##        5.082464        5.207391
\end{verbatim}

\begin{verbatim}
##  Нижняя граница Верхняя граница 
##        5.207391        5.332319
\end{verbatim}

\begin{verbatim}
##  Нижняя граница Верхняя граница 
##        5.332319        5.457246
\end{verbatim}

\begin{verbatim}
##  Нижняя граница Верхняя граница 
##        5.457246        5.582174
\end{verbatim}

\begin{verbatim}
##  Нижняя граница Верхняя граница 
##        5.582174        5.707101
\end{verbatim}

\begin{verbatim}
##  Нижняя граница Верхняя граница 
##        5.707101        5.832028
\end{verbatim}

\begin{verbatim}
##  Нижняя граница Верхняя граница 
##        5.832028        5.956956
\end{verbatim}

\begin{verbatim}
## [1] "Посчитаем теоретические частоты после объединения."
\end{verbatim}

\begin{verbatim}
## [1]  8 11 14 11  6
\end{verbatim}

\begin{verbatim}
## [1] "Частоты в выборке после объединения"
\end{verbatim}

\begin{verbatim}
## [1]  7 12 14 10  7
\end{verbatim}

После всех подсчётов получим 5 теоретических интервалов (после
объединения). \(\chi\)наблюдаемая=0.473 \(\chi\)критическая=5.991 (для
\(alpha\)=0,05, v=2) Вывод: гипотеза о том, что реальная доходность
пенсионных накоплений 50 стран подчиняется нормальному закону
распределения, не отвергается на заданном уровне значимости
\(\alpha\)=0.05

Как мы видим, коэффиценты эксцесса и асимметрии по модулю не превосходят
единицу и достаточно близки к 0. Поэтому можно сделать вывод, что
распределение реальной доходности пенсионных накоплений приблизительно
соответствует нормальному.

Проведём диагностику выбросов. (Можем проводить диагностику, т.к.
проверили распределение на нормальность) 1. С помощью правила 3 сигм:

\begin{verbatim}
##  Нижняя граница Верхняя граница 
##        4.939991        5.956009
\end{verbatim}

Все значения реальной доходности пенсионных накоплений попадают в
указанный интервал.

\begin{enumerate}
\def\labelenumi{\arabic{enumi}.}
\setcounter{enumi}{1}
\tightlist
\item
  Проверим наличие аномальных значений с помощью правила 3IQR(1,5IQR).
\end{enumerate}

\begin{verbatim}
## Нижнняя граница Верхняя граница 
##           5.015           5.855
\end{verbatim}

\begin{verbatim}
## Нижнняя граница Верхняя граница 
##            4.70            6.17
\end{verbatim}

Как мы видим, ни одно значение реальной доходности пенсионных накоплений
не выходит за границы интервалов жёсткого правила (1.5IQR rule) и
мягкого правила (3IQR rule)

3.Построим ящичковую диаграмму:

\begin{verbatim}
## Warning in title(main = "Доходность пенсионных накоплений"): неизвестна ширина
## символа 0xc4
\end{verbatim}

\includegraphics{Answ_files/figure-latex/unnamed-chunk-16-1.pdf}

Получим аналогичный вывод.

\begin{verbatim}
## Warning in title(main = main, sub = sub, xlab = xlab, ylab = ylab, ...):
## неизвестна ширина символа 0xc3
\end{verbatim}

\begin{verbatim}
## Warning in title(main = main, sub = sub, xlab = xlab, ylab = ylab, ...):
## неизвестна ширина символа 0xe8
\end{verbatim}

\begin{verbatim}
## Warning in title(main = main, sub = sub, xlab = xlab, ylab = ylab, ...):
## неизвестна ширина символа 0xf1
\end{verbatim}

\begin{verbatim}
## Warning in title(main = main, sub = sub, xlab = xlab, ylab = ylab, ...):
## неизвестна ширина символа 0xf2
\end{verbatim}

\begin{verbatim}
## Warning in title(main = main, sub = sub, xlab = xlab, ylab = ylab, ...):
## неизвестна ширина символа 0xee
\end{verbatim}

\begin{verbatim}
## Warning in title(main = main, sub = sub, xlab = xlab, ylab = ylab, ...):
## неизвестна ширина символа 0xe3
\end{verbatim}

\begin{verbatim}
## Warning in title(main = main, sub = sub, xlab = xlab, ylab = ylab, ...):
## неизвестна ширина символа 0xf0
\end{verbatim}

\begin{verbatim}
## Warning in title(main = main, sub = sub, xlab = xlab, ylab = ylab, ...):
## неизвестна ширина символа 0xe0

## Warning in title(main = main, sub = sub, xlab = xlab, ylab = ylab, ...):
## неизвестна ширина символа 0xe0
\end{verbatim}

\begin{verbatim}
## Warning in title(main = main, sub = sub, xlab = xlab, ylab = ylab, ...):
## неизвестна ширина символа 0xcd
\end{verbatim}

\begin{verbatim}
## Warning in title(main = main, sub = sub, xlab = xlab, ylab = ylab, ...):
## неизвестна ширина символа 0xb9
\end{verbatim}

\begin{verbatim}
## Warning in title(main = main, sub = sub, xlab = xlab, ylab = ylab, ...):
## неизвестна ширина символа 0xa3
\end{verbatim}

\begin{verbatim}
## Warning in title(main = main, sub = sub, xlab = xlab, ylab = ylab, ...):
## неизвестна ширина символа 0x14
\end{verbatim}

\begin{verbatim}
## Warning in title(main = main, sub = sub, xlab = xlab, ylab = ylab, ...):
## неизвестна ширина символа 0xd7
\end{verbatim}

\begin{verbatim}
## Warning in title(main = main, sub = sub, xlab = xlab, ylab = ylab, ...):
## неизвестна ширина символа 0xb2
\end{verbatim}

\begin{verbatim}
## Warning in title(main = main, sub = sub, xlab = xlab, ylab = ylab, ...):
## неизвестна ширина символа 0x4
\end{verbatim}

\includegraphics{Answ_files/figure-latex/unnamed-chunk-17-1.pdf}

Построим Stemplot:

\begin{verbatim}
## 
##   The decimal point is 1 digit(s) to the left of the |
## 
##   50 | 2
##   50 | 5
##   51 | 111
##   51 | 
##   52 | 11111
##   52 | 7777777
##   53 | 33333333333333
##   53 | 77779
##   54 | 033333333333333
##   54 | 77777777777777
##   55 | 444444444444
##   55 | 688
##   56 | 1114444444
##   56 | 8888
##   57 | 1
##   57 | 999
##   58 | 1
##   58 | 5
\end{verbatim}

Построим точечное распределение:

\begin{verbatim}
## Warning in title(main = main, xlab = xlab, ylab = ylab, ...): неизвестна ширина
## символа 0xd2
\end{verbatim}

\begin{verbatim}
## Warning in title(main = main, xlab = xlab, ylab = ylab, ...): неизвестна ширина
## символа 0xf0
\end{verbatim}

\begin{verbatim}
## Warning in title(main = main, xlab = xlab, ylab = ylab, ...): неизвестна ширина
## символа 0xa3
\end{verbatim}

\begin{verbatim}
## Warning in title(main = main, xlab = xlab, ylab = ylab, ...): неизвестна ширина
## символа 0x14
\end{verbatim}

\begin{verbatim}
## Warning in title(main = main, xlab = xlab, ylab = ylab, ...): неизвестна ширина
## символа 0xd7
\end{verbatim}

\begin{verbatim}
## Warning in title(main = main, xlab = xlab, ylab = ylab, ...): неизвестна ширина
## символа 0xb2
\end{verbatim}

\begin{verbatim}
## Warning in title(main = main, xlab = xlab, ylab = ylab, ...): неизвестна ширина
## символа 0x4
\end{verbatim}

\begin{verbatim}
## Warning in title(main = main, xlab = xlab, ylab = ylab, ...): неизвестна ширина
## символа 0xc7
\end{verbatim}

\begin{verbatim}
## Warning in title(main = main, xlab = xlab, ylab = ylab, ...): неизвестна ширина
## символа 0xed
\end{verbatim}

\begin{verbatim}
## Warning in title(main = main, xlab = xlab, ylab = ylab, ...): неизвестна ширина
## символа 0xe0
\end{verbatim}

\begin{verbatim}
## Warning in title(main = main, xlab = xlab, ylab = ylab, ...): неизвестна ширина
## символа 0xf7
\end{verbatim}

\begin{verbatim}
## Warning in title(main = main, xlab = xlab, ylab = ylab, ...): неизвестна ширина
## символа 0xe5
\end{verbatim}

\begin{verbatim}
## Warning in title(main = main, xlab = xlab, ylab = ylab, ...): неизвестна ширина
## символа 0xed
\end{verbatim}

\begin{verbatim}
## Warning in title(main = main, xlab = xlab, ylab = ylab, ...): неизвестна ширина
## символа 0xe8
\end{verbatim}

\begin{verbatim}
## Warning in title(main = main, xlab = xlab, ylab = ylab, ...): неизвестна ширина
## символа 0xff

## Warning in title(main = main, xlab = xlab, ylab = ylab, ...): неизвестна ширина
## символа 0xff
\end{verbatim}

\includegraphics{Answ_files/figure-latex/unnamed-chunk-19-1.pdf}

Критерий Смирнова-Граббса. Ho-подозрительных наблюдений нет.

\begin{verbatim}
## 
##  Grubbs test for one outlier
## 
## data:  y
## G = 2.52750, U = 0.93482, p-value = 0.518
## alternative hypothesis: lowest value 5.02 is an outlier
\end{verbatim}

p-value=0.1711\textgreater{}0.05. Гипотеза о том, что отсутствует
подозрительное наблюдение принимается на уровне значимости 0.05.

Проведём тест Титьена-Мура. Ho-подозрительных наблюдений нет.
Альтернативная гипотеза- есть 2 подозрительных наблюдения. {[}в этом
тесте не так много смысла, т.к. это модификация критерия Граббса для
нескольких подозрительных наблюдений, но критерий Граббса не показал
наличие подозрительных наблюдений{]}

\(T\) {[}1{]} 0.7707605

\(T\alpha\) 5\% 0.6804644

Как мы видим, наблюдаемое значение больше критического. Гипотеза об
отсутствии подозрительных наблюдений не отвергается на уровне значимости
0.05.


\end{document}
